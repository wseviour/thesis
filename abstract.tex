\begin{abstract}

  Research during the last two decades has established that variability of the
  winter polar stratospheric vortex can significantly influence the troposphere,
  affecting the likelihood of extreme weather events and the skill of long-range
  weather forecasts. This influence is particularly strong following the rapid
  breakdown of the vortex in events known as sudden stratospheric warmings
  (SSWs). This thesis addresses some outstanding issues in our understanding of
  the dynamics of stratospheric variability and its influence on the
  troposphere.

  First, a geometrical method is developed to characterise two-dimensional polar
  vortex variability. This method is also able to identify types of SSW in which
  the vortex is displaced from the pole and those in which it is split in two;
  known as displaced and split vortex events. It shown to capture vortex
  variability at least as well as previous methods, but has the advantage of
  being easily applicable to climate model simulations. Such an application is
  desireable because of the relative lack of SSWs in the observational record.

  % The models display a wide range of biases in the simulated frequency of split
  % and displaced vortex events which are shown to be strongly related to biases
  % in the average state of the vortex

  This method is subsequently applied to 13 stratosphere-resolving climate
  models. Almost all models show split vortex events as barotropic and displaced
  vortex events as baroclinic; a difference also seen in observational
  reanalysis data. This supports the idea that split vortex events are caused by
  a resonant excitation of the barotropic mode. Models show consistent
  differences in the surface response to split and displaced vortex events which
  do not project stongly onto the annular mode. However, these differences are
  approximately co-located with lower stratospheric anomalies. This suggests
  that a local adjustment to stratospheric PV anomalies is the mechanism behind
  the different responses to split and displaced vortex events.

  Finally, the predictability of the polar stratosphere and its influence on the
  troposphere is assessed in a stratosphere-resolving seasonal forecast
  system. Little skill is found in the prediction of the strength of the
  Northern Hemisphere vortex at lead times beyond one month. However, much
  greater skill is found for the Southern Hemisphere vortex during austral
  spring. This allows for forecasts of interannual changes in ozone depletion to
  be inferred at lead times much beyond previous forecasts. It is further
  demonstrated that this stratospheric skill descends with time and leads to an
  enhanced surface skill at lead times of more than three months. 
 



\end{abstract}

%%% Local Variables:
%%% mode: latex
%%% TeX-master: "thesis"
%%% End:
