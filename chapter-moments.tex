\chapter{A geometrical description of vortex variability}
\begin{quotation}
  Much of the work contained in this chapter is based upon the paper
  \citet{Seviour2013a}, although the analysis presented here has been
  significantly extended.
\end{quotation}

\label{cha:moments}

% To Do: 
% - Discussion of sensitivity of the clustering algorithm
% - Description of dripping paint plot
% - Significance for dripping paint plot
% - CP07 and M13 events in table (or maybe disagreeing events highlighed
%   - plot disagreeing events seperately to check
% - Possibly add seasnal cycle of moment diagnostics with percentiles 



\section{Introduction}

A quantitative description of stratospheric polar vortex variability is
desireable for a number of reasons; it allows for the comparison of different
studies, observational data sets, and model simulations, as well as permitting
robust definitions of extreme events.  Traditional methods to quantify vortex
variability have been based on zonal-mean diagnostics, such as the zonal-mean
zonal wind \citep[e.g.,][]{Andrews1987}. This was motivated both by the
simplicity of these diagnostics and the physical reasoning that the strength of
the zonal flow controls the propagation of planetary waves \citep[][Section
\ref{sec:plan-waves-strat}]{Charney1961}. \citet{McInturff1978} provided the
first quantitative definition of SSWs\footnote{In the literature, this is often
  called ``the WMO definition''} (referred to in that text as ``major
stratospheric warmings'') using zonal mean quantities as below.
\begin{quotation}
A stratospheric warming can be said to be major if at 10~mb or below the
latitudinal mean temperature increases poleward from 60 degrees latitude and an
associated circulation reversal is observed (i.e., mean westerly winds poleward
of 60$^{\circ}$ latitude are succeeded by mean easterlies in the same area).
\end{quotation}
A number of variations of this definition have since appeared in the
literature. Most commonly, the temperature gradient criterion has been neglected
and/or zonal wind reversals at a particular latitude (usually $60^{\circ}$N)
used instead of the stricter criterion of a revesal everywhere poleward of
$60^{\circ}$N \citep[e.g.,][]{Labitzke2000, Christiansen2001,
  Reichler2012}. 

Although the reversal of zonal-mean zonal wind is physically relevant for the
propagation of planetary waves, the choice of $60^{\circ}$N and 10~hPa in the
definition of SSWs is less physically significant. Indeed, different numbers of
SSWs are identified if these locations are varied. \citet{Butler2014b} found
that a greater number of events are identified if the threshold is located
either equatorward or poleward of $60^{\circ}$N. Some studies have aimed to
avoid this sensitivity to spatial location by quantifying vortex variability
through empirical orthogonal function (EOF) analysis, using fields over a larger
area. This includes the Northern Annular Mode (NAM) (calculated either from the
three-dimensional geopotential height field \citep{Baldwin2001a} or zonal-mean
geopotential height \citep{Baldwin2009}), EOFs of zonal wind
\citep{Limpasuvan2004}, and vertical profiles of polar cap-averaged temperature
\citep{Kodera2004}. SSW events are then defined by a threshold in the principal
component of the relevant EOF.

As it has become increasingly recognised that SSWs generally occur as either
split or displaced vortex events, studies have aimed to objectively distinguish
these two types of event. Commonly this has been achieved through Fourier
decomposition of the zonal wave structure. For instance, \citet{Nakagawa2006}
defined SSWs through a polar temperature criterion and then split these events
into two groups depending on whether the 150~hPa Eliassen-Palm (EP) flux prior
to the events was dominated by zonal wavenumber one or two. \citet{Charlton2007}
(hereafter CP07) introduced a new classification method, which does not rely on
Fourier decomposition; first they identified events using the traditional wind
reversal at $60^{\circ}$N, 10~hPa criterion, then they calculated the circulation
around the two largest contours of relative vorticity on the vortex edge. If
these two contours have a circulation ratio of 2:1 or lower the event is
classified as a split, and all other events are automatically classed as
displacements. 

Both Fourier decomposition of the zonal wave structure and the method of
\citet{Charlton2007} rely on an Eulerian framework, with fields analysed at a
fixed spatial location. \citet{Waugh1997} first applied two-dimensional moment
diagnostics (otherwise known as elliptical diagostics) to the stratospheric
polar vortex to provide an alternative semi-Lagrangian (or vortex-oriented)
framework. These diagnostics are calculated by fitting an ellipse to a contour
and then determining its properties such as the centre, orientation, aspect
ratio, and area (a further diagnostic, excess kurtosis--a measure of the
`peakednes' of the distribution--was introduced by \citet{Matthewman2009}). This
allows the movement and elongation of the vortex to be
quantified. \citet{Waugh1997} also compared these diagnostics to the traditional
Fourier decomposition. He showed that wave 1 and 2 amplitudes relate most
strongly to the displacement and elongation of the vortex respectively, however,
these relationships were not found to be strong, with correlations of daily
values less than 0.5. These weak relationships were attributed to the fact that
planetary wave propapagtion can be affected by changes in the meridional PV
gradient, even if the vortex shape and location are fixed. Furthermore, the wave
1 amplitude depends to some extent on the elongation of the vortex as well as
the location of the centre (and similarly for the wave 2 amplitude). He
concluded that it is difficult to extract quantitative information about the
shape and location of the vortex based on wave amplitudes alone, highlighting
the advantages of the moment diagnostics.

\citet{Hannachi2010} then applied a heierachical clustering algorithm to daily
values of the area, centroid latitude, and aspect ratio diagnostics and found
that the vortex falls preferably into three clusters corresponding to
undisturbed, split, and displaced states. These groupings were used by
\citet{Mitchell2013} (hereafter M13) to identify split and displaced vortex
events; if the vortex remained in the split or displaced cluster for at least
five consecutive days it was classified as the corresponding
event. Significantly, as discussed in Section \ref{sec:observ-evid}, M13
demonstrated that split vortex events penetrated deep into the troposphere and
resulted in significant surface anomalies, while anomalies associated with
displaced vortex events do not descend far below the tropopause. This is in
agreement with \citet{Nakagawa2006} who found tropospheric anomalies to be
larger following SSWs with dominant wave 2 amplitude, however, it contrasts with
\citet{Charlton2007}, who found little difference in the tropospheric impact of
split and displaced vortex events. This highlights the potential importance of
the method of classification of split and displaced vortex events in any study.

In this chapter we wish to develop a method for the classification of split and
displaced vortex events with the following properties:
\begin{enumerate}
\item Based on vortex moment diagnostics.
\item Can be easily applied to a range of data sets, including climate model
  simulations.
\item Computationally inexpensive. 
\end{enumerate}
The motivation for the use of moment diagnostics includes their advantages in
quantifying the shape and location of the vortex, as noted above. This, in turn,
is desireable because the location of the vortex near the tropopause may be
important for understanting the regional tropospheric effect of stratospheric
anomalies \citep[e.g.,][Section \ref{sec:mechanisms}]{Ambaum2002}. Previous
calculations of vortex moment diagnostics have been based on the distributions
of quasi-conservative tracers such as PV on isentropic surfaces
\citep{Mitchell2011} or long-lived tracer (e.g., N$_{2}$O) concentrations
\citep{Waugh1997}. These quantities have strong meridional gradients allowing
for clear determination of the vortex edge \citep{Nash1996}. Unfortunately, many
climate models do not output PV or tracer concentrations, and these are often
computationally expensive or impractical to calculate. As such, we wish to
develop a method which uses geopotential height, a variable which is output by
all contemporary climate models. This effort will also allow us to test the
robustness of the result of M13 regarding the different surface impacts of split
and displaced vortex events using a semi-independent classification method and
extended data set.

The remainder of this chapter is structured as follows. The next section
introduces the necessary theoretical background for the calculation of moment
diagnostics. Section \ref{sec:methodology} describes the methods used for the
classification of split and displaced vortex events, and compares these events
with those determined by M13 and CP07. Section \ref{sec:moments_analysis}
contrasts the surface impacts of split and displaced vortex events calculated
using the new method and discusses potential mechanisms behind any differences. 


% Stratospheric sudden warmings (SSWs) are extreme events in which the strong
% westerly winds that usually dominate the winter polar stratosphere become highly
% disturbed (here, for reasons outlined below, we use the term SSW to encompass a
% wider range of variability than its traditional definition). These events lead
% to the mixing of mid-latitude air into the polar vortex region, causing an
% increase in temperatures by several tens of kelvin over the course of a few
% days. Traditional methods to identify stratospheric sudden warmings (SSWs) have
% relied on either zonal-mean \citep{Andrews1987} or annular mode
% \citep{Baldwin2001a} diagnostics. Neither method explicitly deals with the
% inherent zonal asymmetry in vortex variability. In particular, SSWs are observed
% to occur in one of two manners: displaced vortex events, where the vortex moves
% far from the pole, and split vortex events, where the vortex separates into two
% `child' vortices. These two types have a very different spatial structure and
% evolution timescale \citep{Matthewman2009}. Displaced and split vortex events
% are predominantly associated with vertically propagating Rossby waves of
% wavenumber 1 and 2 respectively, and many previous studies have classified SSWs
% based on wavenumber \citep[e.g.][]{Nakagawa2006}. However, this method does not
% provide a description of the location of the polar vortex itself, which
% theoretical arguments suggest may be important for understanding
% stratosphere-troposphere coupling \citep{Ambaum2002}. In an improvement to these
% traditional SSW definitions, \citet{Charlton2007a} (hereafter CP07) introduced a
% classification in which a split vortex event is identified when two vortices
% with a circulation ratio of 2:1 or higher are present, and all other SSWs are
% automatically classed as displaced vortex events. However, they maintained the
% traditional SSW identification which requires there to be a reversal of the
% zonal-mean zonal wind at 10~hPa and $60^{\circ}$N.

% An increased understanding of stratospheric variability can be gained by using
% vortex-centric diagnostics, such as two-dimensional (2D) vortex moments
% \citep{Waugh1997, Waugh1999, Mitchell2011, Mitchell2011a}, which provide a
% geometrical description of the vortex and have no reliance on zonal-mean
% properties. Using a classification based on these diagnostics,
% \citet{Mitchell2013} (hereafter M13) identified a greater number of SSWs than
% CP07. This is primarily because they did not use a zonal mean threshold
% criterion. Importantly, M13 also demonstrated that split vortex events
% penetrated deep into the troposphere and resulted in significant surface
% anomalies, while anomalies associated with displaced vortex events do not
% descend far below the tropopause. Their result supported a similar conclusion by
% \citet{Nakagawa2006}, who found that the impact of events associated with an
% enhanced upward flux of wavenumber-2 planetary waves was more likely to reach
% the surface. These results underline the need to correctly identify the precise
% type of SSW, in order to understand stratosphere-troposphere coupling within
% climate models.

% Distinguishing between displaced and split vortex events using the method of M13
% requires the use of potential vorticity (PV), which is not commonly output by
% climate models. For this reason, previous attempts to apply PV-based techniques
% in a multi-model study have led to the majority of models being excluded
% \citep{Mitchell2012a}. Furthermore, their method used a hierarchical clustering
% technique \citep{Hannachi2010}, which is very sensitive to the exact shape of
% the distribution of vortex variability, so is unsuitable for application to a
% range of models with different climatologies. In this chapter, we develop an
% improved method which; (a), is based on the geometry of the vortex, but requires
% only the 10~hPa geopotential height; and (b), identifies events using a simple
% threshold instead of a clustering technique. We apply this new method to the
% ERA-40 and ERA-Interim reanalysis datasets and demonstrate that the method
% captures a similar number of events which are in good agreement with, and at
% least as extreme as, those of M13.

\section{Vortex moment diagnostics}
\label{sec:vort-moment-diagn}

The moments, $M_{n}$, of a one-dimensional distribution can be classified by
their order, $n$, and provide familiar parameters. These are the area under the
distribution (0th order), mean (1st order), variance (2nd order), skewness (3rd
order), and kurtosis (4th order), given by
\begin{equation}
M_{n} = \int_{S} x^{n}f(x)\mathrm{d}x \, ,
\end{equation}
where $S$ represents the extent of the distribution, $f(x)$, to be integrated
over. The extension of this for a two-dimensional distribution is
straightforwardly
\begin{equation}
M_{nm} = \iint_{S} x^{n}y^{m}f(x,y)\mathrm{d}x\mathrm{d}y \, ,
\label{eq:2D_moment}
\end{equation}
where the order of the moment is now defined as $m+n$, meaning it is possible to
have different diagnostics with the same order (e.g., $M_{01}$,
$M_{10}$). Although these diagnostics can be further extended to three
dimensions, this has been demonstrated to be highly computationally expensive
\citep{Li1994}, and would require assumptions about the lower and upper bounds
of the vortex region. We therefore calculate two-dimensional moment diagnostics
for the stratospheric polar vortex on quasi-horizontal surfaces. We use two
variables; geopotential height ($f(x,y) = Z(x,y)$) on the 10~hPa pressure level,
and potential vorticity ($f(x,y) = q(x,y)$) on the 850~K potential temperatre
(isentropic) surface, which lies close to 10~hPa. Following \citet{Waugh1997},
the calculation of moment diagnistics is simplified by tranforming the spherical
data $q(\phi,\lambda)$ and $Z(\phi,\lambda)$, where $\phi$ is latitude and
$\lambda$ longitude, to cartesian coordinates using the polar stereographic
projection
\begin{equation}
x = \frac{\cos\lambda\cos\phi}{1 + \sin\phi}\, , \quad
y = \frac{\sin\lambda\cos\phi}{1 + \sin\phi}\, .
\end{equation}  


In order to calculate moment diagnostics for the stratospheric polar vortex we
must first isolate the vortex region by defining the vortex edge. Different
methods have previously been used for this calculation; \citet{Waugh1999} used
the mean PV at the maximum of the mean meridional PV gradient, while
\citet{Matthewman2009} defined the vortex edge on a daily basis, using the
average value of PV poleward of $45^{\circ}$N nine days before the onset of a
SSW (their SSWs were defined by zonal-mean zonal wind reversal, as in CP07). A
more complex method due to \citet{Nash1996}, starts by transforming PV to
`equivalent latitude' \citep{Butchart1986} coordinates, before defining the
vortex edge as the position of the largest gradient in a plot of PV against
equivalent latitude. This method was applied in \citet{Mitchell2011} to
calculate the vortex edge. 

None of the three methods outlined above are found to be appropriate for the
present study. We wish to directly compare the PV and geopotential
height-derived moments, but the methods of \citet{Waugh1999} and
\citet{Nash1996} rely on meridional gradients in PV and so may not be
transferrable to geopotential height. Furthermore, the method of
\citet{Matthewman2009} is impractical becuase we wish to define the events from
the moment diagnostics, so will not know their dates before
calculation. Instead, we pick a simple definition; PV ($q_{b}$) or geopotential
height ($Z_{b}$) on the vortex edge is defined as the value of the
December-March (DJFM) mean at $60^{\circ}$N. This is seen to lie close to
contours defined by the above methods, and results are insensitive to small
changes in the latitude chosen. 

Having defined the vortex edge, we extend the method \citet{Matthewman2009} to
isolate the vortex region by introducing a transformed PV field, $\hat{q}$,
given by
\begin{equation}
 \hat{q}(x,y) = 
 \begin{cases}
   q(x,y) - q_{b} & \text{if $q(x,y) > q_{b}$} \, , \\
   0 & \text{if $q(x,y) \leq q_{b}$} \, , 
 \end{cases}
\end{equation}
and similarly for geopotential height 
\begin{equation}
 \hat{Z}(x,y) = 
 \begin{cases}
   Z(x,y) - Z_{b} & \text{if $Z(x,y) < Z_{b}$} \, , \\
   0 & \text{if $Z(x,y) \geq Z_{b}$} \, . 
 \end{cases}
\end{equation}
By substituting $f(x,y) = \hat{q}(x,y)$ or $f(x,y) = \hat{Z}(x,y)$ in equation
\ref{eq:2D_moment} it is then possible to calculate the moment diagnostics. The
zeroth order moment diagnostic, $M_{00}$ can be used to define the `equivalent
area', $A_{\mathrm{eq}}$ \citep{Matthewman2009}, as
\begin{equation} 
A_{\mathrm{eq}} = \frac{M_{00}}{q_{b}}\quad \text{or} \quad A_{\mathrm{eq}} = \frac{M_{00}}{Z_b}\, ,
\end{equation}
depending on whether PV or geopotential height based diagnostics are
calculated. Because $M_{00} \approx Aq$, where A is the vortex area, the
equivalent area can be considered a measure of both vortex strength and
area. The first order moment diagnostic can be used to calculate the vortex
centroid,
\begin{equation}
(\bar{x}, \bar{y}) = \left( \frac{M_{10}}{M_{00}}, \frac{M_{01}}{M_{00}} \right)
\, . 
\end{equation}

In order for higher order moment diagnostics to be useful, the moment equation
(\ref{eq:2D_moment}), must be transformed to the \emph{centralised moment}
form \citep{Hall2005}. This calculates moments relative to the vortex centroid,
and is given by
\begin{equation}
J_{mn} = \iint_{S} f(x,y)(x-\bar{x})^n(y-\bar{y})^m\mathrm{d}x\mathrm{d}y \, .
\end{equation}
Two useful parameters can be derived from the second-order centralised moment
diagnostics, the vortex orientation, $\psi$ (defined as the angle between the
major axis of the ellipse and the $x$-axis) and the aspect ratio, $r$ (defined
as the ratio of the lengths of the major to minor axes), given by
\begin{equation}
\psi = \frac{1}{2} \tan^{-1} \left( \frac{2J_{11}}{J_{20}-J_{02}} \right) \, ,
\end{equation}
\begin{equation}
r = \left| \frac{(J_{20}+J_{02})+\sqrt{4J_{11}^2+(J_{20}-J_{02})^2}}
  {(J_{20}+J_{02})-\sqrt{4J_{11}^2+(J_{20}-J_{02})^2}} \right|^{1/2} \, .
\end{equation}
Using the area, centroid, orientation, and aspect ratio, the \emph{equivalent
  ellipse} can be uniquely defined. Figure \ref{fig:displaced_ellipse} shows the
equivalent ellipse calculated from both PV and geopotential height fields over a
16-day period centred on a displaced vortex event (classified using the method
in Section \ref{sec:methodology}). It can be seen that the equivalent ellipse
provides a qualitatively good fit to the vortex, although this is less good in
Figures \ref{fig:displaced_ellipse}(c,f) when the vortex becomes less elliptical
and filamentation occurs. Greater fine-scale structure and filamentation is
visible in the PV field due to its quasi-conservative properties, however
reasonable agreement can be seen between te PV and geopotential height
ellipses. 

\begin{figure}
 \centering
 \noindent\includegraphics[width=\textwidth]{figures/chapter-moments/PV_GPH_2006.pdf}
 \caption[Equivalent ellipse for a displaced vortex event.]{PV on the 850~K
   $\theta$ surface (a,b,c) and geopotential height at 10~hPa (d,e,f) 8 days
   before (a,d), at onset (b,e), and 8 days following the onset (c,f) of a
   displaced vortex event. Contours of $q_{b}$ and $Z_{b}$ are shown in thin
   black lines, the equivalent ellipse in a thick dark line, and its centroid
   with a white cross. Data are transformed to cartesian coordinates with a
   polar stereographic projection.}
 \label{fig:displaced_ellipse}
\end{figure}

Equivalent ellipses for an example of a split vortex event are shown in Figure
\ref{fig:split_ellipse}. It can be seen that after the vortex has separated the
equivalent ellipse becomes less physically significant, as it spans the two
vortices. \citet{Matthewman2009} introduced the 4th order moment diagnostic,
``excess kurtosis'', in order to identify splits of the polar vortex; it is
given by
\begin{equation}
\kappa_4 = M_{00}\frac{J_{40}+2J_{22}+J_{04}}{(J_{20}+J_{02})^2}-\frac{2}{3}\left[\frac{3r^4+2r^2+3}{(r^2+1)^2}\right]\,.
\end{equation}
This has the property of being negative for a vortex with a ``pinched'' shape,
zero for a perfectly elliptical vortex, and positive for a vortex with a strong
central core. When negative kurtosis was detected \citet{Matthewman2009} split
the PV field into two regions along the minor axis of the equivalent ellipse and
re-calculated moment diagnostics for the vortices in these regions separately. 

In this study we do not make use of the excess kurtosis or calculate separate
diagnostics for split vortices for two reasons. First, as a 4th order diagnostic
it is a highly skewed variable, making its use in event classification
problematic (this was also found by \citet{Hannachi2010}). Second, this
procedure is more computationally expensive, requiring about three times the
number of calculations during split vortex events. Hence, we calculate single
moment diagnostics even when the vortex has split, but bear in mind that these
may not represent the properies of any real vortex. 

Code for the calculation of moment diagnostics using the method described
in this section is available from
\url{https://github.com/wseviour/vortex-moments}.

\begin{figure}
 \centering
 \noindent\includegraphics[width=\textwidth]{figures/chapter-moments/PV_GPH_1979.pdf}
 \caption[Equivalent ellipse for a spit vortex event.]{As Figure
   \ref{fig:displaced_ellipse} but for a split vortex event.}
 \label{fig:split_ellipse}
\end{figure}


\section{Data and methods}
\label{sec:methodology}

\subsection{Reanalysis data}

For the analysis in this chapter Northern Hemisphere winter daily-mean data for
December-March (DJFM) are employed from the European Centre for Medium-Range
Weather Forecasts (ECMWF) reanalyses. The ERA-40 data set \citep{Uppala2005} is
used from 1958-1978 and ERA-Interim \citep{Dee2011} from 1979-2009. The
combination of these two data sets is chosen in order to maximise the total
number of years entering the analysis (ERA-40 runs only to 2002), as well as to
compare results from the more recent ERA-Interim with previous studies using
only ERA-40, such as \citet{Charlton2007} and \citet{Mitchell2013}.

ERA-Interim is similar to ERA-40 but uses ECMWF’s operational four-dimensional
variational data assimilation system (4D-Var) as opposed to the 3D-Var system
used in ERA-40. It also has higher horizontal and vertical resolution, improved
humidity analysis, model physics, data quality control, bias handling and other
improvements as noted in \citet{Simmons2007}. The majority of observational data
for the stratosphere entering both reanalyses is from radiosonde and satellite
measurements. It is important to note that in the pre-satellite era (1958-1971)
observations in the stratosphere were much more sparse, leading to greater
errors in reanayses during this time \citep{Uppala2005}. 

A number of studies have evalualted the stratospheric circulation in ERA-40 and
ERA-Interim against other observations or reanalyses. \citet{Randel2004} found
ERA-40 to closely match measurements of the zonal stratospheric circulation
derived from radiosonde, rocketsonde and lidar
measurements. \citet{Karpetchko2005} found that the representation of the polar
vortices in ERA-40 agrees well with the NCEP/NCAR reanalysis, and CP07
demonstrated that this also holds for the occurrence of
SSWs. \citet{Seviour2012} showed that the strength of the stratospheric
meridional mean stratospheric circulation in ERA-Interim agrees well with
previous reanalysis, but that the residual vertical velocity is more smoothly
represented.

In order to perform a consistent analysis across the two data sets, ERA-Interim
data is linearly interpolated to the lower resolution ERA-40
($1.125^{\circ} \times 1.125^{\circ}$) Gaussian grid. PV is also interpolated
from pressure levels to the 850~K isentropic surface (which lies close to
10~hPa), as this quantity has the property of being conserved under adiabatic
flow. In the calculation of the vortex edge. Both in the calculation of the
vortex edge (climatological mean $q$ or $Z$ at $60^{\circ}$N) and the moment
diagnostics themselves, no clear jumps were seen between ERA-40 and ERA-Interim
data sets. As such, the two are considered together with no bias
corrections. For the remainder of this thesis, this combined ERA-40 and
ERA-Interim data set is referred to as `ERA'. 


\subsection{Moment diagnostic calculation}
\label{sec:vort-geom-calc}

In order to calculate the moment diagnostics, the values of PV and geopotential
height on the vortex edge ($q_b$ and $Z_b$) must first be determined. These are
the $60^{\circ}$N December-March mean values of PV at 850~K ($q_{850}$) and
10~hPa geopotential height ($Z_{10}$) respectively. They are found to be
$q_b = 460$~PVU ($\mathrm{1~PVU = 10^{-6}Km^2kg^{-1}s^{-1}}$) and
$Z_b = 30.2$~km. Using these values the moment diagnostics are calculated from
ERA data for DJFM from 1979-2009 using the method described in Section
\ref{sec:vort-moment-diagn}.

As discussed in Section \ref{sec:vort-moment-diagn} the excess kurtosis
diagnostic is not used in the present analysis. In the interests of simplicity,
only the aspect ratio and centroid latitude diagnostics are used to identify
events, and the centroid longitude, orientation and equivalent area are not
used. The aspect ratio and centroid latitude are the most intuitive diagnostics
for this purpose, with a high aspect ratio and poleward centroid latitude
expected during split vortex events, and a low aspect ratio and equatorward
centroid latitude expected during displaced vortex events. 

Figure \ref{fig:pv_z_moments_distribution} shows the distributions of these two
quantities calculated from $q_{850}$ and $Z_{10}$. The centroid latitude
distributions are almost identical, with a peak near $80^{\circ}$N which is in
agreement with previous studies \citep{Mitchell2011,Waugh1999}. The aspect ratio
distributions have a similar shape, with a peak at about 1.3, but the PV based
diagnostic has a larger tail. This is because the PV field contains more
small-scale filamentary structures than geopotential height (e.g. Figures
\ref{fig:displaced_ellipse} and \ref{fig:split_ellipse}), making high aspect
ratios more likely. As well as having similar distributions, the timeseries of
the PV and geopotential height derived diagnostics (not shown) are significantly
correlated, with correlation coefficients of 0.9 for daily centroid latitude and
0.6 for aspect ratio. Overall, these results suggest that geopotential
height-derived moment diagnostics are appropriate for the identification of
split and displaced vortex events. 

\begin{figure}
  \centering
  \includegraphics[width=\textwidth]{figures/chapter-moments/moments_distribution_crop.pdf}
  \caption[Distributions of $Z_{10}$ and PV-based moment
  diagnostics.]{Distributions of the December-March centroid latitude (a) and
    aspect ratio (b), of the Northern Hemisphere stratospheric polar vortex over
    1958-2009. Diagnostics are calculated from geopotential height at 10~hPa
    ($Z_{10}$) and potential vorticity at 850~K (PV). Thresholds of
    66$^{\circ}$N in centroid latitude and 2.4 in aspect ratio are used to
    define events, and are indicated by the black vertical lines.}
  \label{fig:pv_z_moments_distribution}
\end{figure}


\subsection{Event identification}
\label{sec:event-definition}

Previous attempts to identify SSW events have used a clustering method
\citep{K.Coughlin2009,Hannachi2010}. These methods attempt to classify the
vortex state for each day into a number of groups, which may be specified
beforehand or determined by the clustering algorithm. Individual days within the
same cluster should be physically similar, while those in different clusters
distinct. More precisely, clustering aims to maximise the between-cluster
variance while minimising the within-cluster variance. In the case of the
stratospheric polar vortex, clusters may represent, for instance, stable, split,
and displaced states. Events are then typically defined by the vortex persisting
in a particular cluster for a number of days.

\begin{figure}
 \centering
 \noindent\includegraphics[width=0.7\textwidth]{figures/chapter-moments/clustering.pdf}
 \caption[Clustering algorithms applied to the moment diagnostics.]{Three
   clustering algorithms and a threshold division applied to the moment
   diagnostics in centroid latitude-aspect ratio space. For the $k$-means and
   hierarchical algorithms three clusters were specified. The mean-shift
   algorithm determined the number of clusters to be 4.}
 \label{fig:clusters}
\end{figure}

A large number of clustering algorithms exist, and some may be more appropriate
than others for certain uses. Here, three different algorithms are applied to
the moment diagnostics in centroid latitude-aspect ratio space, and their
outcomes shown in Figure \ref{fig:clusters}(a,b,c). Details of the three
algorithms are given below:
\begin{enumerate}[(a)]
\item \textbf{\textit{K}-means} clustering requires the number of clusters, $K$,
  to be specified beforehand (in Figure \ref{fig:clusters}, $K=3$). The
  algorithm begins by randomly selecting $K$ data points to be the centroids of
  the initial clusters, all other data points are assigned to the cluster with
  the nearest centroid. Having assigned the initial cluster membership, the
  algorithm proceeds as follows:
  \begin{enumerate}[1.]
    \item Compute the centroids (the vector means), $\mathbf{\overline{x}}_k$
      of each cluster. 
    \item Calculate the distance between the current data point, $\mathbf{x}_i$,
      and each of the $K$ $\mathbf{\overline{x}}_k$s. (Various distance measures
      can be used; in Figure \ref{fig:clusters}, the Euclidean distance is used).
    \item If $\mathbf{x}_i$ is not in the group with the closest mean then
      reassign it to that group, otherwise repeat step 2 for $\mathbf{x}_{i+1}$.
  \end{enumerate}
  This is repeated until a full cycle through each $\mathbf{x}_i$ produces no
  reassignments. An advantage of this method is that it is computationally
  efficient, but the major disadvantage is that the number of clusters must be
  pre-determined. Several methods exist to estimate the ideal number of
  clusters, which generally have the aim of finding the best comprimise between
  minimising within-cluster variance and maximising between-cluster
  variance. \citet{K.Coughlin2009} applied $K$-means clustering to several
  variables representing the stratospheric polar vortex. They used a method
  known as silhuette values \citep{Rousseeuw1987} and determined the ideal
  number of clusters to be two (representing stable and disturbed vortex
  states).

\item \textbf{Mean-shift} clustering aims to discover `blobs' in a data set. It
  works by updating candidates for centroids to be the mean of the points within
  a given region. That is, given a candidate centroid $\mathbf{x}_i$ for
  iteration $t$, the candidate is updated according to
  \begin{equation}
    \mathbf{x}^{t+1}_{i} = \mathbf{x}^t_i + \mathbf{m}(\mathbf{x}^t_i) \, ,
  \end{equation}
 where $\mathbf{m}$ is the mean shift vector. This is calculated as
 \begin{equation}
  \mathbf{m}(\mathbf{x}_i) = \frac{\sum_{\mathbf{x}_j \in N(\mathbf{x}_i)}K(\mathbf{x}_j-\mathbf{x}_i)\mathbf{x}_j}{\sum_{\mathbf{x}_j
      \in N(\mathbf{x}_i)}K(\mathbf{x}_j-\mathbf{x}_i)} \, ,
 \end{equation}
 where $K(\mathbf{x}_j-\mathbf{x}_i)$ is a kernel function which determines the
 weight of nearby points. Typically, and in Figure \ref{fig:clusters}, a
 Gaussian kernal is used,
 $K(\mathbf{x}_j-\mathbf{x}_i) = e^{-c\|\mathbf{x}_j-\mathbf{x}_i\|^2}$.
 $N(\mathbf{x}_i)$ represents the set of points for which
 $K(\mathbf{x}_i) \ne 0$. This shifting is repeated until $m$
 converges. Following this calculation, the candidates are then filtered to
 remove near duplicates. The greatest advantage of this method is that it
 automatically sets the number of clusters, so no prior assumptions about the
 data set are required. A disadvantage is that it requires multiple nearest
 neighbour searches during each iteration, and so may not be scalable to large
 data sets.



\item \textbf{Hierarchical}
\end{enumerate}



In order to identify displaced and split vortex events, a threshold criterion is
introduced and applied to the geopotential height derived diagnostics. A
displaced vortex event requires the centroid latitude to remain equatorward of
66$^{\circ}$N for 7 days or more. A split vortex event requires the aspect ratio
to remain higher than 2.4 for 7 days or more. These thresholds are indicated in
Figure \ref{Fig1}, and were selected to give a similar frequency of displacement
and splitting events as M13. This choice is somewhat subjective, but the results
presented below are not sensitive to the exact choice of threshold. There were
no occasions on which both criteria were met simultaneously. The onset date is
defined as the day that the appropriate threshold is first exceeded, and to
ensure that no events are counted twice, the events are required to be spaced at
least 30 days apart, chosen to reflect radiative timescales in the lower
stratosphere \citep{Newman1997}.  \citet{Mitchell2011} found that the aspect
ratio and centroid latitude follow an extreme value distribution \citep{Cole}
and we note that both thresholds chosen here lie beyond the extreme value
thresholds of their respective distributions. Using this method, 17 displaced
and 18 split vortex events (listed in Table 2.1) are identified over the 52
winters, an average of 7 per decade. These events are all mid-winter events --
the DJFM data were used explicitly to avoid counting final warmings. This
frequency lies between the values of CP07 (6/decade) and M13 (8/decade).

 \begin{figure}
 \centering
 \noindent\includegraphics[width=\textwidth]{figures/chapter-moments/splits_displacements_histogram.pdf}
 \caption[Seasonal distribution of dispalaced and split vortex
 events.]{Histogram of the seasonal distribution of displaced and split vortex
   events, form the $Z_{10}$ method, M13 and CP07.}
 \label{Fig2}
 \end{figure}
 There are significant differences in the seasonal distribution of displaced and
 split vortex events, as shown in Figure \ref{Fig2}. Split vortex events are
 more frequent in early-mid winter, with a peak in January, while displaced
 vortex events are skewed towards late winter. Figure \ref{Fig2} also shows the
 seasonal distribution of displaced and split vortex events from CP07 and
 M13. The shape of the distribution agrees well with the seasonal distribution
 found by M13 for both types of event. However, there is less similarity with
 the CP07 distribution of displaced vortex events (Figure \ref{Fig2}b). CP07
 indicates an approximately flat distribution throughout winter, and many fewer
 displaced vortex events overall. It should be noted that the seasonal
 distribution of displaced and split vortex events from the moment based methods
 does not arise from the underlying climatology of centroid latitude or aspect
 ratio, which remains approximately constant throughout winter
 \citep{Mitchell2011}.

\begin{table}
\begin{centering}
    \begin{tabular}{|l|l|l|l|l|}  \hline
    No. & Event onset & Event type & $\Delta \mathrm{T}_{10}$ (K) &
                                                                    $\overline{U}_{10} (\mathrm{m~s^{-1}})$ \\ \hline
    1*  & 1961-3-9    & D          & 10.2       & 2.7 \\
    2*  & 1962-1-30   & S          & 1.9        & 38.9 \\
    3*  & 1962-3-7    & S          & -1.0       & 16.9 \\
    4*  & 1964-3-15   & D          & 11.9       & 1.3 \\
    5   & 1966-2-26   & D          & 2.5        & -5.9 \\
    6   & 1967-12-29  & S          & 13.0       & 19.4 \\
    7   & 1970-1-5    & S          & 8.5        & -4.0 \\
    8   & 1971-1-15   & S          & 10.8       & -1.7 \\
    9*  & 1972-2-4    & S          & -1.6       & 33.6 \\
    10  & 1973-2-4    & S          & 7.3        & -6.6 \\
    11* & 1974-3-12   & D          & 5.3        & -4.8 \\
    12* & 1975-3-16   & D          & 7.6        & -8.0 \\
    13* & 1976-3-31   & D          & 8.2        & -13.3 \\
    14  & 1977-1-7    & S          & 7.6        & -5.5 \\
    15* & 1978-3-25   & D          & 2.5        & -9.3 \\
    16  & 1979-2-18   & S          & 5.6        & -0.4 \\
    17  & 1984-2-25   & D          & 11.6       & -4.4 \\
    18  & 1984-12-25  & S          & 15.0       & -1.7 \\
    19* & 1986-1-7    & S          & 3.4        & 29.9 \\
    20* & 1986-3-21   & D          & 9.1        & -12.2 \\
    21  & 1987-1-20   & D          & 8.3        & -7.7 \\
    22  & 1987-12-10  & S          & 9.8        & -3.0 \\
    23* & 1992-3-22   & D          & 7.6        & -4.4 \\
    24* & 1995-2-2    & D          & 5.6        & 7.7 \\
    25  & 1998-12-15  & D          & 8.2        & 8.1 \\
    26  & 1999-2-24   & S          & 6.6        & -12.7 \\
    27  & 2001-2-7    & S          & 5.2        & -7.2 \\
    28* & 2001-3-15   & S          & -6.8       & 12.1 \\
    29* & 2002-3-21   & S          & -1.5       & 5.1 \\
    30  & 2003-1-17   & S          & 6.1        & 16.8 \\
    31  & 2004-1-2    & D          & 5.8        & -4.8 \\
    32* & 2005-3-11   & D          & 3.1        & -5.0 \\
    33  & 2006-1-17   & D          & 4.2        & -14.3 \\
    34  & 2008-2-18   & D          & 4.6        & 2.3 \\
    35  & 2009-1-18   & S          & 13.2       & 16.9 \\ \hline
    \end{tabular}
    \caption{A summary table of displaced (D) and split (S) vortex events,
      identified using the Z10 method for data from 1958-2009.
      $\Delta \mathrm{T}_{10}$ represents the mean area-weighted
      $50^{\circ}$-$90^{\circ}$N cap temperature anomaly at 10 hPa calculated 5
      days either side of the event onset date. $\overline{U}_{10}$ represents
      $\overline{U}$ at $60^{\circ}$N and 10~hPa averaged over the same
      period. Stars (*) represent those numbers that do not coincide (within 10
      days) with events defined by CP07.}
\end{centering}
\label{tab:events}
\end{table}

% Mean values 
% -----------
% DT: Split: 5.7 pm 5.7 (3.0, 8.3)
%     Displ: 6.8 pm 2.9 (5.5, 8.2)
%     Split (CP07): 8.6 pm 4.6 (6.1, 10.9)
%     Displ (CP07): 7.8 pm 3.9 (5.5, 9.9)

\section{Analysis}
\label{sec:moments_analysis}

To evaluate how well the new method captures displaced and split vortex events,
Figure \ref{Fig3} shows composites of PV in the mid-stratosphere (850~K)
following their onset dates. These are averaged over the 5 days following the
onset date for split events and 7 days for displaced events (these averaging
periods reflect the different timescales of the events). The composites are
compared with the corresponding composites following the events identified by
M13 and CP07. For the split vortex events, the $Z_{10}$ method clearly shows two
separated vortices, one centred over Canada and the other over Siberia. This
characteristic direction reflects the climatological wave-2 pattern at this
altitude. For the M13 events the split vortex composite shows the vortex
stretched across the same $90^{\circ}$W-$90^{\circ}$E line, although not as
clearly split, while the composite for the CP07 events looks very
different. This has a weak vortex centred over Canada, with the other over
Northern Europe, and is similar to the composite for displaced events. All three
composites for displaced events show a vortex centred over Northern Europe, but
this extends most westward in the CP07 composite, suggesting that there may be
some contamination from misdiagnosed split vortex events.
\begin{figure}
 \centering
 \noindent\includegraphics[width=\textwidth]{figures/chapter-moments/pv_composites_colbar_crop.pdf}
 \caption[PV composites for split and displaced vortex events.]{Composites of
   potential vorticity at the 850~K isentropic surface from the ECMWF reanalyses
   over 1958-2009. Composites are taken over the 5 days following the onset date
   of split vortex events (a,b,c) and 7 days following displaced vortex events
   (d,e,f) (the difference is due to the different timescales of these
   events). We compare the current ($Z_{10}$) method (a,b) with that of M13
   (b,e) and CP07 (c,f).}
 \label{Fig3}
 \end{figure}

 Figure \ref{Fig3} demonstrates that the $Z_{10}$ method succeeds in identifying
 displaced and split vortex events events as well as, and in some cases better
 than, the methods of M13 or CP07. When comparing the three methods, CP07 is the
 clear outlier. This is most likely because the CP07 approach employs a
 zonal-mean threshold which cannot accurately capture the extreme events (M13).

\section{Summary}

Recent research has demonstrated the need to distinguish between displaced and
split stratospheric polar vortex events, due to their different impacts on
surface weather patterns. However, current methods to identify these events are
complex or require non-standard variables. In this chapter, a new, robust method
has been developed to identify displaced and split vortex events, which requires
only geopotential height at 10~hPa. The method is briefly summarised as follows:
 \begin{enumerate}
 \item To identify the vortex region, a single contour of 10~hPa geopotential
   height is selected: the value of the zonal mean at $60^{\circ}$N.
 \item Using this contour and the geopotenial height field, the centroid
   latitude and aspect ratio are calculated, following the methodology of
   \cite{Matthewman2009}.
 \item Events are identified using a threshold criterion: Displaced events are
   identified if the centroid latitude remains equatorward $66^{\circ}$N for 7
   days or more. Split events are identified if the aspect ratio remains above
   2.4 for 7 days or more. No two events may occur within 30 days.
 \end{enumerate}

 Results using this method demonstrate that it is able to identify split and
 displaced vortex events at least as effectively as previous methods.

% ------------------- Copied from 2nd year report --------------%
\pagebreak



\begin{figure}
 \centering
 \noindent\includegraphics[width=\textwidth]{figures/chapter-moments/dripping_paint.png}
 \caption[NAM composites for split and displaced vortex events.]{Composites of
   the time-height evolution of the NAM during (a) 17 vortex displacement events
   and (b) 18 splitting events. Lag 0 shows the onset of an event as measured at
   10 hPa. Contour intervals are 0.25 and the region between -0.25 and 0.25 is
   unshaded. Data is from the ECMWF Reanalyses 1958-2009.}
 \label{fig:dripping_paint}
\end{figure}

\begin{figure}
 \centering
 \noindent\includegraphics[width=0.7\textwidth]{figures/chapter-moments/nam_difference_sig.pdf}
 \caption[Significance of surface NAM difference following split and displaced
 vortex events.]{Distribution of 0-30 day mean NAM composite differences between
   split and displaced vortex events, formed by randomly shuffling the labels
   `split' and `displacement' between events. The 95\% significant region
   (according to a two-tailed test) is shaded and the true composite difference
   is at the 94th percentile.}
 \label{Fig3}
\end{figure}

\begin{figure}
 \centering
 \noindent\includegraphics[width=0.7\textwidth]{figures/chapter-moments/mslp_composites_colbar_crop.pdf}
 \caption[Mean sea-level pressure composites for split and displaced vortex
 events.]{Composites of mean sea-level pressure anomalies in the 30 days before
   (a,b) and 30 days after (c,d) the onset dates of displaced (a,c) and split
   (b,d) vortex events from the $Z_{10}$ method. Data is from the ECMWF
   renalyses (1958-2009). Anomalies are calculated for each day and gridpoint
   from the climatology for that day of the year and gridpoint. Grey contours
   indicate regions of greater than 95\% statistical significance according to a
   Monte-Carlo significance test.}
 \label{fig:mslp_composites}
\end{figure}

\begin{figure}
 \centering
 \noindent\includegraphics[width=0.8\textwidth]{figures/chapter-moments/tropopause_height_composites_nam_crop.png}
 \caption[Tropopause height composites for split and displaced vortex
 events.]{Composites of tropopause height anomalies averaged 10 days before
   (a,b), 10 days after (c,d) and 10-20 days after displaced and split vortex
   events. Anomalies are calculated for each day and gridpoint from the
   climatology for that day of the year and gridpoint. Stippling indicates
   regions of greater than 95\% statistical significance according to a
   Monte-Carlo significance test. Grey contours indicate the first EOF of NH
   mean sea-level pressure, which explains 33\% of the variance.}
 \label{fig:tropopause_height}
\end{figure}



M13 found that the mean sea-level pressure (MSLP) anomalies are different before
and after displaced and split vortex events. In Figure \ref{fig:mslp_comp} we
present composites of MSLP 30 days before and 30 days following the onset dates
of displaced and split vortex events, calculated using the $Z_{10}$ method, from
the ECMWF reanalyses. Statistical significance is estimated from a Monte-Carlo
method, using $10^{5}$ composites of equal size, formed from randomly sampled
winter dates. The strongest precursor is found for displaced vortex events, with
a wave-1 pattern that is similar to the stationary wave pattern
\citep[e.g.][]{Garfinkel2008}, suggesting increased wave-1 propagation into the
stratosphere. However, the strongest anomalies following events occur after
split vortex events, with a pattern resembling the negative phase of the North
Atlantic Oscillation (though with a southern centre of action shifted towards
Europe).

The evolution of anomalies throughout the depth of the atmosphere can be
investigated through the use of annular modes \citep[e.g.][]{Baldwin2001a}. The
Northern Annular Mode (NAM) (known as the Arctic Oscillation at the surface) is
the leading mode of variability in the wintertime variability of the Northern
Hemisphere circulation. Here we use the method of \citep{Baldwin2009}, who
define the NAM as the leading empirical orthogonal function (EOF) of daily
wintertime (November-April) zonal mean geopotential height anomalies poleward of
$20^{\circ}$N. The anomalies are calculated by subtracting the seasonal cycle
which has been smoothed with a 90-day low-pass filter. The daily NAM anomalies
are then determined by projecting daily geopotential anomalies onto the leading
EOF patterns. Finally, the NAM is normalised at each level so that the entire
time series has unit variance.

Figure \ref{fig:nam_comp} presents time-height composites of NAM anomalies 90
days either side of the onset date of displaced and split vortex events. Despite
the larger stratospheric anomaly following displaced vortex events, the
tropospheric signal is larger following split vortex events (as in Figure
\ref{fig:mslp_comp}). The vertical evolution of these events also differs
greatly, with anomalies descending from the upper stratosphere over a period of
weeks for displacement events, while split vortex events are near
barotropic. This suggests an excitation of the barotropic mode, which supports
the idea of the wave resonance view of SSWs \citep{Esler2005}.

Overall, these results support the findings of M13, but use a new method to
identify split and displaced vortex events, and extend the analysis to 2009. The
fact that the results are consistent suggests the findings of M13 are robust and
highlights the importance of distinguishing split and displaced vortex events in
understanding stratosphere-troposphere coupling following SSWs.






%%% Local Variables:
%%% mode: latex
%%% TeX-master: "thesis"
%%% End:







