\chapter*{List of acronyms and abbreviations}
\addcontentsline{toc}{chapter}{List of acronyms and abbreviations}

\begin{description*}

\item[CCMVal-2] Second Chemistry--Climate Validation Activity 
\item[CFC] Chlorofluorocarbon
%\item[CMCC] Centro Euro-Mediterraneo per I Cambiamenti Climatici
\item[CMIP5] Coupled Model Intercomparison Project Phase 5
\item[CP07] \citet{Charlton2007}
%\item[DJF] December to February (inclusive)
\item[ECMWF] European Centre for Medium-Range Weather Forecasts
\item[ENSO] El Ni\~no--Southern Oscillation
\item[EOF] Empirical orthogonal function
\item[EP] Eliassen-Palm
\item[ERA] Combination of ERA-40 reanalysis (1958--1978) and ERA-Interim (1979--2010)
\item[GCM] General circulation model
\item[GEV] Generalised extreme value
\item[GloSea5] Met Office Global Seasonal Forecast System 5
%\item[GFDL] Geophysical Fluid Dynamics Laboratory
%\item[IPSL] Institut Pierre-Simon Laplace
\item[LOOCV] Leave-one-out cross-validation
\item[M13] \citet{Mitchell2013}
%\item[MIROC]
\item[MMM] Multi-model mean
%\item[MPI] Max Planck Institute
%\item[MRI] Meteorological Reseach Institute
\item[MSLP] Mean sea-level pressure
\item[NAM] Northern annular mode
\item[NAO] North Atlantic oscillation
\item[PNA] Pacific-North American (pattern)
\item[PSC] Polar stratospheric cloud
\item[PV] Potential vorticity
%\item[\textbf{\textit{q$_{850}$}}] PV on the 850~K potential temperature surface
\item[QBO] Quasi-biennial oscillation
\item[QG] Quasi-geostrophic
\item[ROC] Reveiver operating characteristic
\item[SAM] Southern annular mode
\item[SH] Southern Hemisphere
\item[SOI] Southern oscillation index
%\item[SON] September to November (inclusive)
\item[SSW] Sudden stratospheric warming
\item[TEM] Transformed Eulerian-mean
\item[UTLS] Upper-troposphere/lower-stratosphere
%\item[\textbf{\textit{Z$_{10}$}}] 10~hPa geopotential height 

\end{description*}



%%% Local Variables:
%%% mode: latex
%%% TeX-master: "thesis"
%%% End:
