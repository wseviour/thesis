\chapter{Conclusions}
\label{cha:conclusions}

\section{Summary of results}

The main findings of this thesis are summarised as follows: 
\begin{itemize}
\item\textbf{Application of moment diagnostics:} It has been demonstrated that
  vortex moment diagnotsics can be successfully applied to the geopotential
  height field, giving similar results as when applied to conservative fields
  such as PV. This therefore provides a semi-Lagrangian (or vortex-centric)
  method which can be readily used to describe the geometry of the stratospheric
  polar vortex in climate model simulations.

  It has been further shown that a simple threshold-based method can be applied
  to the vortex moment diagnostics in order to indentify split and displaced
  vortex events. The events identified in this way coincide to a large extent
  with events defined by other methods, and capture equally extreme vortex
  states. 

\item\textbf{The stratospheric polar vortex in climate models:} The first
  multi-model comparison of stratospheric polar vortex geometry and split an
  displaced vortex events has been carried out using the stratosphere-resolving
  CMIP5 models. A wide range of biases have been identified in the geometry of
  the stratospheric polar vortex among models. Some models have a vortex which
  is on average too equatorward, others too poleward, while the majority of
  models have a vortex which is too circularly symmetric. Models also vary
  widely in their frequency of split and displaced vortex events. However, the
  nature of these events is largely in agreement with observations, in
  particular the fact that split vortex events appear more barotropic and
  displaced vortex events baroclinic. The consistency of this difference in
  baroclinicity among models lends weight to the idea that split vortex events
  are caused by a resonant excitation of the barotropic mode, as suggested by
  \citet{Esler2005}. Significantly, the frequency of split and displaced vortex
  events has been demonstated to be highly correlated respectively with the
  aspect ratio and centroid latitude of the average undisturbed vortex. It
  therefore follows that an improvement in the mean state of the vortex is
  likely to lead to a more accurate representation of these extremes.

\item\textbf{Stratosphere-troposphere coupling in climate models and
    observations:} In reanalysis data, using the geopotential height-based
  vortex moments method, a stronger tropospheric NAM signal is seen following
  split vortex events than displaced vortex vortex events. This is in agreement
  with the results of \citet{Mitchell2013}. However, a bootstrap significance
  test of the surface NAM over the month following these events cannot exclude
  the possibility that this observed difference is due to chance. 

  In the CMIP5 models, the tropospheric NAM signal following both split and
  displaced vortex events is weak on average. There is no consistent difference
  between the two apart from close to the onset of events when there is a
  negative anomaly for split vortex events which extends barotropically through
  the depth of the atmosphere. However, looking at two-dimensional tropospheric
  anomalies in mean sea-level pressure following split and displaced vortex
  events shows some consistent features. A negative NAO-like signal is seen
  which is of similar magnitude following both types of event. The Pacific
  response is much less robust, with some models simulating negative pressure
  anomalies, and others positive. The discrepancy between the Atantic and
  Pacific responses suggests that the annular mode may not be a good metric for
  stratosphere-troposphere coupling in the NH.

  Almost all models show more negative sea-level pressure anomalies over Siberia
  following displaced vortex events than split vortex events. Overall, the
  differences in the surface signals following the two types of events are
  approximately co-located with the difference in lower-stratospheric
  geopotential height, which in turn follow stratospheric PV anomalies. A
  similar pattern is also seen in tropopause height in reanalysis data. This
  suggests the mechanism behind the different surface responses to split and
  displaced vortex events is one local to lower stratospheric PV anomalies, as
  proposed by \citet{Ambaum2002}. However, it should be stressed that the
  similarities in the NAO response suggest that other mechanisms more sensitive
  to zonal-mean anomalies, such as baroclinic instability or planetary wave
  reflection, also play a role. 

\item\textbf{Predictability of the polar stratosphere:} Using hindcast
  simulations produced by a stratosphere-resolving seasonal forecast system, no
  skill has been found in the prediction of NH SSWs or split or displaced vortex
  events at lead times beyond one month. This suggests that the skillful
  seasonal prediction of the winter NAO in the same system \citep{Scaife2013} is
  not highly influenced by the stratosphere. It may, however, be attributable to
  other model improvements such as increased atmospheric and oceanic horizontal
  resolution.

  On the other hand, skillful prediction of the SH stratospheric polar vortex
  during the austral spring at seasonal lead times has been found. This skill is
  greater than a persistence forecast; indeed, a strong late-summer polar vortex
  is related to a weak spring vortex, indicating the importance of
  preconditioning. Using the observed relationship between the strength of the
  stratospheric polar vortex and polar ozone, it was possible to produce
  skillful forecasts of interannual variations in polar stratospheric ozone
  depletion. This prediction is at longer lead times than previous
  forecasts. Furthermore, because interannual variability is significant when
  compared to the long-term ozone depletion trend, such forecasts may be of some
  interest for populations in the SH.

  A further feature of the hindcast simulations is that the year 2002, in which
  the only observed SH SSW occurred, is also the most extreme of the hindcasts
  with almost all ensemble members simulating negative stratospheric wind
  anomalies. It also has one of the two out of 210 ensemble members which
  simulate SH SSW-like events (although these are displaced vortex events,
  rather than the split that occurred). This suggests that an increased
  likelihood of the 2002 event may have been detectable almost two months in
  advance. 

\item\textbf{Stratospheric influence on tropospheric predictability:}

\end{itemize}

\section{Extensions of this work}

\begin{itemize}
\item\textbf{Decadal variability:}

\item\textbf{Systematic variation of model resolution:}

\item\textbf{Time scale of stratosphere-troposphere coupling:}

\item\textbf{More hindcast simulations:}

\item\textbf{The 2002 Southern Hemisphere SSW:}

\item\textbf{Influence of interactive chemistry on seasonal forecast skill:}

\end{itemize}

%%% Local Variables:
%%% mode: latex
%%% TeX-master: "thesis"
%%% End:
