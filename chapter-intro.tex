\chapter{Introduction}
\label{cha:intro}

\section{Overview and aims}
%original work


Traditionally the stratosphere was thought to respond passively to tropospheric
forcing from below. However, modelling and observational evidence accrued over
the last two decades has demonstrated that in some cases variability of the
winter polar stratosphere can cause consistent circulation anomalies at the
Earth's surface. This influence has been shown to be particularly strong
following the rapid breakdown of the usual westerly winter stratospheric polar
vortex; events known as sudden stratospheric warmings (SSWs).

Despite these advances, important issues remain as to the dynamics of SSWs
and the stratosphere's influence on the troposphere. Most significantly:
\begin{enumerate}[i.]
\item The dynamics of SSWs are not fully understood, in particular whether
  different mechanisms may be responsible for different types of event.
\item A mechanism for the stratosphere's influence on the troposphere is not
  well developed and it is not understood why some types of stratospheric event
  appear to have different impacts on the troposphere than others.
\end{enumerate}
These are significant long-standing issues, and providing a comprehensive
solution is not possible here, but it is hoped that this thesis will go some way
to addressing them. A solution to these issues is not purely of theoretical
interest, since it is necessary to understand the dynamics of these phenomena in
order to simulate them in weather and climate prediction models; the ultimate
aim being the development of more skillful forecasts. With an eye on this
application, this thesis also aims to assess the predictability of the
stratosphere and its connection to the troposphere.

The main original contributions of this thesis to the scientific literature are
summarised below:
\begin{enumerate}[1.]
\item In \textbf{Chapter \ref{cha:moments}} a new method to diagnose
  stratospheric polar vortex variability and classify split and displaced vortex
  events is introduced and tested. This is the first semi-Lagrangian (or
  vortex-centric) method that can be easily and robustly applied to climate
  model simulations. Reanalysis data are then used to compare anomalies at the
  tropopause and the surface following the split and displaced vortex
  events. Although there may be some significant differences, the relatively
  short observational record hinders the statistical significance of these
  results.

\item In \textbf{Chapter \ref{cha:models}} this method is applied to carry out
  the first multi-model comparison of split and displaced vortex events and
  their influence on the troposphere. It is found that there is a wide range of
  biases in the representation of the stratospheric polar vortex among models,
  at least some of which may be attributable to differences in vertical
  resolution. It is also shown that there are consistent differences between the
  tropospheric response to split and displaced vortex events among the
  models. These differences, and the large number of events studied, allows some
  inference of the mechanisms behind the different surface responses.

\item In \textbf{Chapter \ref{cha:seas}} the predictability of the stratospheric
  polar vortex is assessed in a stratosphere-resolving seasonal prediction
  system. Little skill is found in the prediction of the strength of the
  Northern Hemisphere stratospheric polar vortex or the occurrence of split or
  displaced vortex events on seasonal timescales. However, significant skill is
  found in the case of the Southern Hemisphere vortex. This enables the skillful
  prediction of interannual variability in ozone depletion beyond the lead time
  of previous forecasts. Furthermore, it is demonstrated that the stratospheric
  skill significantly enhances the skill of tropospheric forecasts several
  months ahead.

\end{enumerate}

\section{Relation to published work}
Chapter \ref{cha:moments} is largely based on a paper written by myself, Daniel
Mitchell and Lesley Gray in \emph{Geophysical Research Letters}
\citep{Seviour2013}, although the analysis has been significantly extended and
re-written. The results in Chapter \ref{cha:seas} on the Southern Hemisphere are
based on a paper written by myself, Steven Hardiman, Lesley Gray, Neal Butchart,
Craig MacLachlan, and Adam Scaife in \emph{Journal of Climate}
\citep{Seviour2014}. 

In both of these papers, all the writing is my own and I carried out all the
analysis and produced the figures. However, I am of course very grateful for the
constructive comments of my coauthors in the preparation of these papers as well
as my reviewers; Harry Hendon (from the Centre for Australian Weather and
Climate Research), and three of whom are anonymous.

\section{Thesis structure}

The next chapter introduces the necessary background of the current
understanding of the dynamics of the polar stratosphere, including the
differences between the Northern and Southern Hemispheres and SSWs. It also
reviews the role of dynamics in polar stratospheric ozone depletion, the
atmospheric annular modes, and the observational, modelling, and theoretical
evidence for the stratosphere's influence on the troposphere. The original
results described above are presented in Chapters \ref{cha:moments},
\ref{cha:models}, and \ref{cha:seas}. Conclusions and possible extensions to
the work in this thesis are discussed in Chapter \ref{cha:conclusions}. 



%%% Local Variables:
%%% mode: latex
%%% TeX-master: "thesis"
%%% End:







