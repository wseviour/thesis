\chapter{Introduction}
\label{cha:introduction}

\section{Overview and aims}
%original work
The stratosphere is the second layer of the atmosphere above the Earth's
surface, bounded by the tropopause below and the stratopause above. The height
of the tropopause varies from about 15~km in altitude in the tropics to 7~km at
high latitudes, while the stratopause lies at approximately 50~km. The defining
feature of the stratosphere is a temperature gradient increasing with height (in
contrast to the troposphere below), caused by the presence of ozone which
absorbs ultraviolet radiation\footnote{For this reason, the region of the
  stratosphere with the highest ozone concentrations is often called the ``ozone
  layer''.}. This temperature gradient makes the stratosphere stable against
vertical convection and results in very different dynamical behaviour to the
troposphere.

Traditionally, the stratosphere was thought to respond passively to tropospheric
forcing from below. However, modelling and observational evidence accrued over
approximately the last two decades has suggested that the variability of the
winter polar stratosphere can cause consistent circulation anomalies at the
Earth's surface. Stratosphere-troposphere coupling has been found to be
particularly strong following the rapid breakdown of the 

Despite these advances, many questions remain as to the
stratosphere's influence on the troposphere. Firstly, it is not understood why
some types of stratospheric event appear to have a larger impact on the
stratosphere than others


\subsection{Relation to published work}
Chapter \ref{cha:moments} is largely based on a paper written by myself, Daniel
Mitchell and Lesley Gray in \emph{Geophysical Research Letters}
\citep{Seviour2013}, although the analysis has been significantly extended and
re-written. The results in Chapter \ref{cha:seas} on the Southern Hemisphere are
based on a paper written by myself, Steven Hardiman, Lesley Gray, Neal Butchart,
Craig MacLachlan, and Adam Scaife in \emph{Journal of Climate}
\citep{Seviour2014}. 

In both of these papers, all the writing is my own and I carried out all the
analysis and produced the figures. However, I am of course very grateful for the
constructive comments of my coauthors in the preparation of these papers as well
as my reviewers; Harry Hendon (from the Centre for Australian Weather and
Climate Research), and three of whom are anonymous.



\section{Dynamics of the polar stratosphere}

\subsection{Zonal mean flow}

Each winter the polar regions descend into a polar night and the stratosphere
cools by infrared radiation to space. This sets up a strong meridional
temperature gradient which increases the vertical zonal wind shear in accordance
with the thermal wind balance relation
\begin{equation}
\frac{\partial u_g}{\partial z} = -\frac{R}{fH}\frac{\partial T}{\partial y} \,, 
\end{equation} 
where $u_g$ is the geostrophic zonal velocity,
$u_g = -f^{-1}\partial Z/\partial y$, $Z$ is geopotential height, $f$ is the
Coriolis parameter, $f=2\Omega\sin\phi$, and $R$ is the specific gas
constant. Here, a beta-plane geometry is used such that $f=f_0+\beta y$, where
$f_0=f(\phi_0)$ and $\beta = 2\Omega a^{-1}\cos\phi_0$, $a$ is the Earth's
radius and $\phi_{0}$ is a reference latitude. $H$ is the scale height given by
$H = RT_s/g$, where $T_s$ is a reference temperature and $g$ is the acceleration
due to gravity. This equation relies on hydrostatic and geostrophic
approximations but is approximately satisfied on seasonal timescales. Hence, the
meridional temperature gradient decreasing from equator to pole results in a
region of westerly winds surrounding the pole; this is known as the
stratospheric polar vortex.

\begin{figure}
 \centering
 \noindent\includegraphics[width=\textwidth]{figures/chapter-intro/zmzw_zmT_clim.pdf}
 \caption[Zonal-mean zonal wind and temperature climatology.]{December-Janurary
   (DJF) (a,c) and July-August (JJA) (b,d) averages of zonal-mean zonal wind
   ($\mathrm{m~s^{-1}}$) (a,b) and temperature (K) (c,d). Data is from the
   ERA-Interim reanalysis (1979-2010).}
 \label{fig:zmzw_zmT_clim}
\end{figure}

Figure \ref{fig:zmzw_zmT_clim} shows zonal-mean zonal wind and temperature
averaged over the boreal winter (December-February; DJF) and austral winter
(July-August; JJA) using data from 1979-2010 from the ERA-Interim reanalysis
(details in Section \ref{sec:reanalysis-data}). In both cases the westerly
vortex in the winter hemisphere can be seen along with a local minimum in
temperature at the winter pole in the lower stratosphere. Weaker easterly winds
are present in the summer hemisphere. The maximum strength of the polar vortex
occurs at midlatitudes between 0.1-1~hPa in the mesosphere, and is stronger in
the Southern Hemisphere (SH) with a maximum of $90~\mathrm{ms^{-1}}$ than the
Northern Hemisphere (NH) with a maximum of $50~\mathrm{ms^{-1}}$. The winter
polar stratosphere is also approximately 20~K colder in th SH than the NH. 

\begin{figure}
 \centering
 \noindent\includegraphics[width=\textwidth]{figures/chapter-intro/zmzw_NH_SH.pdf}
 \caption[Comparison of NH and SH polar vortex seasonal cycle.]{Seasonal cycle
   of NH (a) and SH (b) polar vortex strength, measured by $\overline{u}$ at
   60$^{\circ}$N/S, 10~hPa. The annual mean is shown in a thick black line and
   individual years in thin grey lines. Both time series are centred on their
   respective winters. Data is from the ERA-Interim reanalysis (1979-2009).}
 \label{fig:zmzw_NH_SH}
\end{figure}

The maximum strength of the vortex in the stratosphere occurs at approximately
$60^{\circ}$N/S with little variation through the depth of the
stratosphere. Figure \ref{fig:zmzw_NH_SH} shows the annual cycle and variability
of zonal-mean zonal wind at 10~hPa $60^{\circ}$N and $60^{\circ}$S. As well as
being weaker on average than the SH, the winter NH stratospheric polar vortex
can also be seen to be significantly more variable than the SH. There are a
number of years in the NH for which $\overline{u}$ becomes negative during the
winter, but only one such year in the SH (these events are discussed further in
Section \ref{sec:strat-sudd-warm}). A further clear feature of both NH and SH is
that variability during the summer is much less than that during winter. All
these observations are due almost entirely to the influence of wave phenomena in
the stratosphere, as described in the next section.


\subsection{Waves in the stratosphere}
\label{sec:plan-waves-strat}

\subsubsection{Planetary waves}

Large-scale Rossby or planetary waves\footnote{Here, as is common in the
  stratospheric literature, ``wave'' is taken to mean any deviation from the
  zonal-mean state which not necessarily a physical structure.} play a vital role
in the dynamics of the extratropical stratosphere. They mostly enter the
stratosphere from the troposphere, where they are forced, for example, by air
flow around large-scale orography or air-sea temperature contrasts. There
large-scale waves approximately satisfy the quasi-geostropic (QG) approximation
of hydrostatically balanced incompressible flow with low Rossby number,
$\mathrm{Ro} = U/f_oL \ll 1$, where $U$ and $L$ are characteristic velocity and
length scales respectively \citep{Andrews1987}. Under this approximation and in
the absence of friction, the following relation, known as the
\emph{quasi-geostrophic potential vorticity equation}, holds:
\begin{equation}
D_gq_g = f_0\rho_0 \frac{\partial}{\partial z}
\frac{\rho_0Q}{\partial\theta_{0}/\partial z} \, . 
\label{eq:qgpv}
\end{equation}
Where
\begin{equation}
D_g \equiv \frac{\partial}{\partial t} + u_g\frac{\partial}{\partial x} +
v_g\frac{\partial}{\partial y} \, , 
\end{equation}
and 
\begin{equation}
  q_g = f_0 + \beta y - \frac{\partial v_g}{\partial x} + \frac{\partial u_g}{\partial
    y} + \rho_o^{-1}\frac{\partial}{\partial
    z}\left(\rho_of_0\frac{\theta_e}{\partial\theta_{0}/\partial z}\right) \, , 
\end{equation}
is the quasi-geostrophic potential vorticity. Here, $v_g$ is the geostrophic
meridional velocity, $v_g = f^{-1}\partial Z/\partial x$, $Q$ is the diabatic
heating rate, $\rho_0$ is a reference density and $\theta_0$ is a reference
potential temperature, $\theta_0 = T_s(p_s/p)^\kappa$, where
$p_s=1000~\mathrm{hPa}$, and $\kappa = R/c_p \approx 2/7$, where $c_p$ is the
specific heat capacity of air at constant pressure. $\theta_e$ represents the
departure from $\theta_0$, and is assumed to be small in the sense that
$|\partial\theta_e/\partial z| \ll |\partial\theta_0/\partial z|$. An important
consequence of Equation \ref{eq:qgpv} is that $q_g$ is conserved following the
geostrophic wind for adiabatic flow ($Q=0$), and therefore acts as a tracer.

In the case of approximately zonal flow $[\overline{u}(y,z),0,0]$, Equation
\ref{eq:qgpv} can be linearised to give
\begin{equation}
\left(\frac{\partial}{\partial t} + \overline{u}\frac{\partial}{\partial
    x}\right)q_g' + v'\frac{\partial\overline{q_g}}{\partial y} = f_0\rho_0 \frac{\partial}{\partial z}
\frac{\rho_0Q'}{\partial\theta_{0}/\partial z} \, ,
\label{eq:linear_qgpv} 
\end{equation}
where primes represent deviations from the zonal mean (e.g., $q_g =
\overline{q_g} + q_g'$). It can be shown that Equation \ref{eq:linear_qgpv}
supports wave-like solutions, with vertical propagation dependent upon the
condition:
\begin{equation}
0 < \overline{u}-c < \overline{u}_c \equiv \beta(k^2+l^2+\epsilon/4H^2)^{-1} \,,
\end{equation}
which is known as the \emph{Charney-Drazin criterion} after
\citet{Charney1961}. Here, $c$ is the wave's zonal phase speed, $k$ and $l$ are
the zonal and meridional wavenumbers respectively, and $\epsilon = f_0^2/N^2$,
where $N^2 = H^{-1}Re^{-\kappa z/H}\partial\theta_0/\partial z$ is the static
stability. In the case of waves whose phase is stationary with respect to the
ground ($c=0$), this simplifies to
\begin{equation}
0 < \overline{u} < \overline{u}_c\, . 
\label{eq:charney-drazin}
\end{equation}
It is therefore apparent that in order for planetary waves to propagate
vertically (such as from the troposphere to the stratosphere), a westerly flow
must be present that is not too strong. Additionally, this maximum speed is
dependent on wavenumber, such that a lower wavenumber can propagate in a
stronger westerly flow. While the assumptions here are not representative of the
real atmosphere (such as purely zonal flow, and small deviations from the zonal
mean), this criterion does capture the most important features of the relation
between zonal assymmetries and the zonal flow, and similar relations can be
found for more complex background states \citep{Andrews1987}.

An important consequence of the Charney-Drazin criterion for stratospheric flow
is that the strength of the stratospheric polar vortices shown in Figures
\ref{fig:zmzw_zmT_clim} and \ref{fig:zmzw_NH_SH} is often sufficient to exclude
all but the lowest wavenumbers (typically zonal wavenumbers 1 and 2) from
propagating upwards from the troposphere. Hence the length-scale of typical
stratospheric zonal assymmetries is much larger than that of the troposphere. 

\bigskip When planetary waves reach a \emph{critical surface}, where propagation
is prohibited (for instance, a region where $\overline{u} = c$), the above
linear analysis breaks down. In this scenario waves can ``break'', imparting
momentum onto the zonal flow. There is therefore a two-way interaction between
the zonal flow and planetary waves; a phenomenon known as \emph{wave-mean flow
  interaction}. Wave breaking was studied in an idealised two-dimensional model
by \citet{Stewartson1978} and \citet{Warn1978}. They found momentum to be
absorbed in a narrow \emph{critical layer} close to the critical surface, with
potential vorticity (PV) contours being irreversibly stretched and mixed in
anticyclonic structures known as ``Kelvin's cats' eyes''. They also showed that
the critical layer is initially absorbing, but becomes a reflecting surface
after some time. Time varying results from a version of the Stewartson-Warn-Warn
model from \citet{Andrews1987} are shown in Figure \ref{fig:cats_eyes}. Similar
wave breaking behaviour as this idealised model was first observed in the real
stratosphere by \citet{McIntyre1983}.

\begin{figure}
 \centering
 \noindent\includegraphics[width=0.5\textwidth]{figures/chapter-intro/breaking_wave_AHL.png}
 \caption[Results from a Stewartson-Warn-Warn model of wave breaking.]{}
 \label{fig:cats_eyes}
\end{figure}

A further effect of wave breaking is the induction of a \emph{residual
  circulation}, $[0, \overline{v}^*, \overline{w}^*]$, where $\overline{v}^*$ and
$\overline{w}^*$ are the transformed Eulerian-mean (TEM) meridional and vertical
velocities given by
\begin{equation}
  \overline{v}^* = \overline{v} - \frac{1}{\rho_0}\frac{\partial}{\partial z}
  \left(\frac{\rho_o\overline{v'\theta'}}{\partial\overline{\theta}/{\partial z}}\right) \, , 
\end{equation}
\begin{equation}
\overline{w}^* = \overline{w} + \frac{1}{a\cos\phi}\frac{\partial}{\partial\phi}
\left(\frac{\cos\phi\overline{v'\theta'}}{\partial\overline{\theta}/{\partial
      z}}\right) \, ,
\end{equation}
which approximate the Lagrangian-mean circulation under time-averaged conditions
\citep{Andrews1976,Dunkerton1978,Holton1980a}. Under the TEM formalism, the zonal
momentum equation becomes
\begin{equation}
\frac{\partial\overline{u}}{\partial t} +
\overline{v}^*\left(\frac{1}{a\cos\phi}\frac{\partial}{\partial\phi}(\overline{u}\cos{\phi})
  - f \right) + \overline{w}^*\frac{\partial\overline{u}}{\partial z} =
\frac{\mathbf{\nabla \cdot F}}{\rho_oa\cos\phi} + \overline{X}
\label{eq:zonal_momentum}
\end{equation}
where $\overline{X}$ represents frictional terms and $\mathbf{F}$ is the
Eliassen-Palm (EP) flux with components
\begin{equation}
F^\phi = \rho_0a\cos\phi\left(\frac{\partial\overline{u}}{\partial
    z}\frac{\overline{v'\theta'}}{\partial\overline{\theta}/\partial z} -
  \overline{v'u'}\right) \, ,
\end{equation}
\begin{equation}
F^z =
\rho_0a\cos\phi\left(\left[f-\frac{1}{a\cos\phi}\frac{\partial}{\partial\phi}(\overline{u}\cos\phi)\right]\frac{\overline{v'\theta'}}{\partial\theta/\partial
      z} - \overline{w'u'}\right) \, .
\end{equation}
$\mathbf{F}$ can be interpreted as the flux of wave activity
\citep{Andrews1987}, and therefore $\mathbf{\nabla\cdot F}<0$ (covergence)
represents a dissipation of wave activity, as is the case in wave breaking. It
can be seen that for a steady zonal flow ($\partial\overline{u}/\partial t = 0$)
in the absence of wave driving ($\mathbf{\nabla\cdot F} = 0$) or friction
($\overline{X} = 0$), a solution of Equation (\ref{eq:zonal_momentum}) is
$\overline{v}^*=0, \overline{w}^*=0$.  However, in the presence of these forcing
terms, a non-zero residual circulation will be induced. Climatologically, this
circulation consists of upwelling in the tropics and poleward and downward
motion in the extratropics; a pattern known as the Brewer-Dobson circulation. It
is observed that this circulation is strongest in the winter hemisphere due to
the fact that more planetary waves can propagate and break in the winter
westerly flow than the summer easterly flow (due to the Charney-Drazin
criterion). Furthermore, during periods of enhanced wave breaking the residual
circulation accelerates and there is more decent and adiabatic heating at high
latitudes. This is important in the physical understanding of sudden
stratospheric warming events, described in Section \ref{sec:strat-sudd-warm}.


\subsubsection{Gravity waves}

Gravity waves are another type of atmospheric wave important in the dynamics of
the polar stratosphere. These waves owe their existence to buoyancy restoring
forces and can be generated by a number of processes such as air flow over
orography (orographic gravity waves), convection or frontogenesis
(non-orographic gravity waves). As with planetary waves, the differences in the
land masses of the two hemispheres leads to orographic gravity wave activity
being much greater in the NH. These waves make a net easterly contribution to
the winter zonal flow \citep[e.g.,][]{Seviour2012}, and so act to enhance the
residual circulation. Their typical length scales are much shorter than can be
resolved in general circulation models or reanalyses, and so they are usually
perameterised, appearing as the term $\overline{X}$ in the zonal momentum
equation (Equation \ref{eq:zonal_momentum}).

\bigskip Together, the Charney-Drazin criterion and the effects of planetary and
gravity wave driving on the zonal flow can expain almost all hemispheric
differences seen in Figures \ref{fig:zmzw_zmT_clim} and \ref{fig:zmzw_NH_SH}:
Greater orography results in more planetary and gravity wave generation in the
NH, both of which cause a net deceleration of the westerly polar vortex, thereby
causing the NH vortex to be weaker than the SH. This also explains why the NH
vortex is warmer than the SH, as the greater NH wave activity induces a stronger
residual circulation with enhanced descent and adiabatic warming at high
latitudes. Additionally, the strength of the SH vortex is such that it prohibits
the vertical propagation of planetary waves from the troposphere throughout much
of the winter, meaning that the SH vortex is less variable than the NH. Both
hemispheres show very little variability in the summer easterly flow because
planetary wave propagation is prohibited in this regime.


\subsection{Sudden stratospheric warmings}
\label{sec:strat-sudd-warm}
%footnote on name
First observed by \citet{Scherhag1952} in radiosonde measurements over Berlin,
the extreme events visible in Figure \ref{fig:zmzw_NH_SH} whereby the winter
circulation temporarily becomes easterly\footnote{For a discussion of more
  precise definitions of SSWs, see Section \ref{sec:moments-introduction}.} are
known as sudden stratospheric warmings\footnote{Following \citet{Butler2014a} it
  is suggested that the term ``sudden stratospheric warming'' is preferrable to
  the common alternative ``stratospheric sudden warming''. This is because there
  are other varieties of stratospheric warming (such as final warmings or
  Canadian warmings), but not other varieties of atmospheric sudden warming.}
(SSWs). These events occur approximately 5-7 times per decade in the NH, but
only one such event has been observed in the $\sim 60$ year observational record
in the SH (in 2002). They are called ``warmings'' because associated with the
circulation reversal is a dramatic increase in temperature; as much as 50~K in
the space of a few days in the midstratosphere.

Initially these events were thought to result from either solar storms or
baroclinic instability of the stratospheric polar vortex. However,
\citet{Matsuno1970, Matsuno1971} proposed a model of SSWs which relies on the
influence of tropospherically forced planetary waves. This model (or
modifications thereof) remains the most widely accepted dynamical view of SSWs
at present. The mechanism proceeds as follows:
\begin{enumerate}[i.]
\item A packet of enhanced planetary wave activity enters the stratosphere where
  it reaches a critical surface and breaks. This decelerates the zonal flow over
  a broader critical layer, and if strong enough causes it to reverse.
\item Hence a new, lower critical surface is formed (where $\overline{u}=0$),
  and wave breaking occurs at this level. The process continues as the critical
  layer descends to the lower stratosphere. 
\item At the same time, wave breaking induces an enhanced residual circulation
  with greater descent and adiabatic warming at high latitudes. If strong
  enough, this can act to reverse the meridional temperature gradient, further
  enhancing the easterly flow by thermal wind balance. 
\item When the critical surface is close to the tropopause, planetary wave
  activity is essentially prohibited from entering the polar
  stratosphere. Radiative cooling to space then acts to cool the polar
  stratosphere and the vortex reforms over a period of approximately 2-4 weeks.
\end{enumerate}

This mechanism takes place in an essentially zonal-mean framework. However, it
has been observed that SSWs generally occur as either a split or displacement of
the vortex, mostly depending (though not exlusively; see Section
\ref{sec:moments-introduction}, \citep{Waugh1997}) on whether wave-2 or wave-1
activity is dominant. An example of each of these events is shown in Figure
\ref{fig:charlton_polvani_ssw}. \citet{Charlton2007a} and \citet{Matthewman2009}
studied the dynamics of these two types of events in reanalysis data and noted
some differences. Most significantly, split vortex events where observed to
occur near-barotropically, with two smaller vortices centred over Canada and
Siberia throughout the depth of the stratosphere. On the other hand, displaced
vortex events were observed to be more baroclinic, starting first in the upper
stratosphere with a vortex centred over Canada, the centre of which rotates
westward with height and is centred over Siberia in the lower stratosphere.


\begin{figure}
 \centering
 \noindent\includegraphics[width=\textwidth]{figures/chapter-intro/charlton_polvani_SSW.png}
 \caption[Examples of a split and displaced vortex event from
 \citet{Charlton2007a}]{Polar stereographic plot of geopotential height
   (contours) on the 10~hPa pressure surface. Contour interval is 0.4~km, and
   shading shows potential vorticity greater than
   $4.0 \times 10^{-6} \mathrm{K~kg^1~m^2~s^{-1}}$. (a) A vortex displacement
   type warming that occurred in February 1984. (b) A vortex splitting type
   warming that occurred in February 1979. From \citet{Charlton2007}.}
 \label{fig:charlton_polvani_ssw}
\end{figure}

This different behaviour of split and displaced vortex events is not accounted
for by the \citet{Matsuno1971} model above, and so may suggest that other
mechanisms are important in the generation of SSWs. For instance,
\citet{ONeill1988} and \citet{Scott2006} have suggested that SSWs can be
generated by a cyclone-anticyclone interaction between the Aleutian high and the
polar vortex. These studies showed that a smaller anticyclone can act to
significantly distort the polar vortex, although such interactions are greatest
for circulation ratios higher than are typically found in the polar
stratosphere. 

Other studies have suggested that SSWs can arise through the resonant excitation
of normal modes of the stratosphere \citep{Tung1979, Plumb1981} 


\section{Stratospheric ozone depletion}
\section{Stratosphere-troposphere coupling}

\begin{figure}
 \centering
 \noindent\includegraphics[width=\textwidth]{figures/chapter-intro/Baldwin_Dunkerton.png}
 \caption[NAM composite from \citet{Baldwin2001a}]{Composites of time-height
   development of the northern annular mode for (A) 18 weak vortex events and
   (B) 30 strong vortex events. The events are determined by the dates on which
   the 10-hPa annular mode values cross –3.0 and +1.5, respectively. The indices
   are nondimensional; the contour interval for the color shading is 0.25, and
   0.5 for the white contours. Values between −0.25 and 0.25 are unshaded. The
   thin horizontal lines indicate the approximate boundary between the
   troposphere and the stratosphere. From \citet{Baldwin2001a}.}
 \label{fig:baldwin_dunkerton}
\end{figure}


\subsection{Observational evidence}
\label{sec:observ-evid}
\subsection{Modelling evidence}
\subsection{Mechanisms}
\label{sec:mechanisms}
% Look at Song and Robinson 2004 for basis of mechanisms review
% Radiative Grise

\section{Thesis plan}

\begin{figure}
 \centering
 \noindent\includegraphics[width=0.5\textwidth]{figures/chapter-intro/mean_tropopause_height.pdf}
 \caption[]{ }
 \label{fig:cmip5_mslp_diff}
\end{figure}



%%% Local Variables:
%%% mode: latex
%%% TeX-master: "thesis"
%%% End:
