\chapter{Introduction}
\label{cha:intro}

\section{Overview and aims}
%original work
The stratosphere is the layer of the Earth's atmosphere which lies above the
troposphere and is bounded by the tropopause below and the stratopause
above. The height of the tropopause varies from about 15~km in altitude in the
tropics to 7~km at high latitudes, while the stratopause lies at approximately
50~km. The defining feature of the stratosphere is a temperature gradient
increasing with height (in contrast to the troposphere below), caused by the
presence of ozone which absorbs ultraviolet radiation\footnote{For this reason,
  the region of the stratosphere with the highest ozone concentrations is often
  called the ``ozone layer''.}  and thereby heats the surrounding
atmosphere. This temperature gradient makes the stratosphere stable against
vertical convection and results in very different dynamical behaviour to the
troposphere.

Traditionally the stratosphere was thought to respond passively to tropospheric
forcing from below. However, modelling and observational evidence accrued over
approximately the last two decades has suggested that in some cases variability
of the winter polar stratosphere can cause consistent circulation anomalies at
the Earth's surface. This influence has been shown to be particularly strong
following the rapid breakdown of the usual westerly winter stratospheric polar
vortex; events known as sudden stratospheric warmings (SSWs). 

Despite these advances, important issues remain as to the dynamics of SSWs
and the stratosphere's influence on the troposphere. Most significantly:
\begin{enumerate}[i.]
\item The dynamics of SSWs are not fully understood, in particular whether
  different mechanisms may be responsible for different types of event.
\item A mechanism for the stratosphere's influence on the troposphere is not
  well developed and is not understood why some types of stratospheric event
  appear to have a larger or different impact on the troposphere than others.
\end{enumerate}
These are large and long-standing issues, and providing a comprehensive solution
is not possible here, but it is hoped that this thesis will go at least some way
to addressing them. A solution to these issues is not purely of theoretical
interest, as it is necessary to understand the dynamics of these phenomena in
order to model them and put them to use in weather and climate predictions. With
an eye on this application, we also aim to assess the predictability of the
stratosphere and its connection to the troposphere in these models.

The main original contributions of this thesis to the scientific literature are
outlined below:
\begin{enumerate}[1.]
\item In \textbf{Chapter \ref{cha:moments}} a new method to diagnose
  stratospheric polar vortex variability and classify split and displaced vortex
  events is introduced and tested. This is the first vortex-centric method that
  can be easily and robustly applied to climate model simulations.

\item In \textbf{Chapter \ref{cha:models}} this method is applied to carry out
  the first multi-model comparison of split and displaced vortex events and
  their influence on the troposphere. It is found that there are a wide range of
  biases among models, at least some of which may be attributable to differences
  in vertical resolution. Furthermore, the large number of events studies allows
  some inference of the mechanisms behind the different surface response to
  split and displaced vortex events. 

\item In \textbf{Chapter \ref{cha:seas}} the preidctability of the stratospheric
  polar vortex is assessed in a stratosphere-resolving seasonal prediction
  system. Little skill is found in the prediction of the strength of Northern
  Hemisphere stratospheric polar vortex or the occurrence of split or displaced
  vortex events on seasonal timescales. However, sigificant skill is found in
  the case of the Southern Hemisphere vortex, allowing for the prediction of
  interannual variability in ozone depletion and with a significant influnce on
  the surface beyond the lead time of previous forecasts. 
\end{enumerate}

\section{Relation to published work}
Chapter \ref{cha:moments} is largely based on a paper written by myself, Daniel
Mitchell and Lesley Gray in \emph{Geophysical Research Letters}
\citep{Seviour2013}, although the analysis has been significantly extended and
re-written. The results in Chapter \ref{cha:seas} on the Southern Hemisphere are
based on a paper written by myself, Steven Hardiman, Lesley Gray, Neal Butchart,
Craig MacLachlan, and Adam Scaife in \emph{Journal of Climate}
\citep{Seviour2014}. 

In both of these papers, all the writing is my own and I carried out all the
analysis and produced the figures. However, I am of course very grateful for the
constructive comments of my coauthors in the preparation of these papers as well
as my reviewers; Harry Hendon (from the Centre for Australian Weather and
Climate Research), and three of whom are anonymous.

\section{Thesis structure}


%%% Local Variables:
%%% mode: latex
%%% TeX-master: "thesis"
%%% End:







